\section{Introduction}
\label{sec:intro}

Many commercially deployed systems, such as Amazon Dynamo\cite{decandia2007dynamo}, sacrifices strong consistency properties for high availability and throughput. This sacrifice is driven by the need of providing client applications with an ``always-on" experience. For example, a customer needs to manipulate items in the shopping cart without any latency, but it is usually acceptable to the customer if deleted items appear again due to conflicts or failures unless it is not the final check-out. Another example may be a user trying to change his profile picture. Strong consistency is not required here because the user usually allows some time for the change to take effect. However, when the user place the order, strong consistency is highly preferred. 

CRAQ\cite{terrace2009object} is a chain replication storage system which guarantees strong consistency, but it targets at read-mostly workloads and cannot provide high write throughput/availability because every write request needs to go through the head of the chain. Apparently, given inexpensive hardware, one can never guarantee strong consistency and high write throughput at the same time. Based on this assumption, we ask the question: in terms of the performance, can we flexibly switch between CRAQ and Dynamo? In other word, can we design a storage system which still supports $put()$ and $get()$ operations, but $put()$ accepts one more parameter - $consistency$? $consistency$ is a binary flag indicating whether the object being updated is strongly consistent or eventually consistent. 

We call such a system as CRAQamo, implying that it is a hybrid of CRAQ and Dynamo. Specifically, when the client application does not require high availability, for example, the shopping cart, CRAQamo only supports eventual consistency. When the client application requires strong consistency, for example, the customer's list of orders, CRAQamo sacrifices write throughput to guarantee strong consistency. CRAQamo is not a system which simply runs CRAQ and Dynamo in parallel, because each of these two systems have their own arrangements of nodes, failure detection and recovery strategies. CRAQamo merges them into a single system for maximal efficiency.

